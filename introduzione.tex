\chapter{Introduzione}
\label{chap:intro}

Considerato che, negli ultimi anni, lo sviluppo e controllo di software aziendali si sta spostando in maniera decisa verso il Web, permettendo una maggior portabilitá e scalabilitá, con conseguente controllo da remoto tramite qualsiasi dispositivo connesso alla rete e che l’utilizzo di dispositivi di input che utilizzano il protocollo HID é sempre piú richiesto e talvolta necessario per una miglior esperienza utente, il gruppo Loccioni ci ha proposto di analizzare le possibili strade da percorre per integrare un dispositivo Microsoft, in particolare il Surface Dial, che grazie al posizionamento su schermo touch-screen, permetta l’acquisizione di componenti e il controllo degli stessi.
Lo scopo finale di questo progetto è quindi quello poter integrare il dispositivo “Microsoft Surface Dial” nei software Loccioni, sia lato Web che Windows Form, con l'obiettivo di rendere piú fluida e intuitiva l’esperienza utente, consentendo di svolgere operazioni in parallelo che, normalmente vengono eseguite tramite dispositivi di input classici come mouse e tastiera, attraverso l’utilizzo di entrambe le mani, mantenendo il focus di un dispositivo su un determinato compito.

Aggiungere foto banchi test loccioni e allungare il brodo.

\section{Obiettivo}
Descrivere brevemente la visione di Luca per l'utilizzo finale del dial nei banchi test Loccioni

\section{Struttura della Tesi}
L'esposizione di questo elaborato sará cosi suddivisa:
- il capitolo n "nome" descrive bla bla bla
- il capitolo n "nome" descrive bla bla bla
- il capitolo n "nome" descrive bla bla bla
- il capitolo n "nome" descrive bla bla bla