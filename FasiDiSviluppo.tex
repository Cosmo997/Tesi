\chapter{Fasi Di Sviluppo}
\label{chap:fasi}

In questo capitolo verranno dettagliate le fasi atte allo sviluppo del progetto che é stato possibile suddividere in 4 \emph{macro fasi}:

\begin{itemize}
\item \textbf{Prima Fase}: Analisi e Studio di fattibilità.
\item \textbf{Seconda Fase}: Implementazione comunicazione bi-direzionale tra UWP e Web.
\item \textbf{Terza Fase}: Ottimizzazione della complessità nella comunicazione.
\item \textbf{Quarta Fase}: Creazione Servizio Angular e implementazione posizionamento sullo schermo.
\end{itemize}

\section{Prima Fase}
Nella prima fase, durata circa un mese e mezzo, ci siamo dedicati alla scoperta delle tecnologie utilizzate e alla fattibilità delle richieste presentate dall’Impresa. Per prima cosa ci siamo dedicati allo studio del Microsoft Dial e alle sue potenzialità, scoprendo le librerie dedicate ad esso e le applicazioni che le supportavano.
\\

La prima importante decisione è stata quale piattaforma utilizzare per l’avvio del progetto, è la decisione è poi ricaduta sulle applicazioni UWP per 3 motivi principali:
\begin{enumerate}
\item Il continuo rilascio da parte di Microsoft di miglioramenti su questo tipo di piattaforma, dato che diventerà la base di tutti i programmi dell’Impresa che utilizzano il Dial.
\item La documentazione scritta ed i progetti d’esempio che utilizzano il Dial sono tutti fatti attraverso UWP.
\item ’applicazione UWP consente l'utilizzo al suo interno di una WebView che ci permette di comunicare con la pagina Web.
\end{enumerate}


Questa prima fase è stata molto impegnativa, poichè la documentazione presente è basilare e non spiega interamente come svolgere le azioni più complesse, ma siamo riusciti a portarla a termine anche grazie al nostro collega Sebastiano Verdolini, che ha svolto il suo stage universitario con noi.

\section{Seconda Fase}

Nella seconda fase, durata circa 5 settimane, abbiamo sviluppato prototipi di applicazioni UWP atti a testare i vari comportamenti del Dial e le sue funzionalitá native, integrando verso la fine una pagina HTML statica all'interno della WebView, includendo delle funzioni JavaScript per la comunicazione UWP $\rightarrow$ Web.\\

Oltre a questo abbiamo seguito un \textbf{Webinar} su Angular e sul framework Aulos, creato ed utilizzato, dall'impresa stessa.
Nella parte conclusiva della fase siamo riusciti ad utilizzare un decoratore di classe chiamato \textbf{AllowForWeb} presente in UWP che ci ha permesso di iniettare a RunTime in una nostra pagina Web, un oggetto C\# che rappresenta il ponte di comunicazione tra le due piattaforme in maniera \emph{bi-direzionale}.


\section{Terza Fase}

Nella terza fase abbiamo iniziato a lavorare sul GetStarted fornitoci dal gruppo Loccioni con i servizi Aulos da loro sviluppati.Una volta compreso il funzionamento del Framework Aulos ci siamo spostati sull'integrazione della pagina Web nella nostra WebView al fine di poterla controllare grazie agli sviluppi conseguiti durante la precedente fase.\\

Dopo vari tentativi e difficoltà, siamo riusciti ad integrarla all'interno della nostra UWP, ma rimaneva il problema principale riscontrato in questa fase, ovvero il posizionamento all'interno della parte Web delle funzioni richiamabili dalla UWP.
Grazie all'aiuto dei nostri tutor siamo arrivati alla soluzione che ci consente di inserire una classe con i relativi metodi messi a disposizione all'interno della variabile globale "\emph{window}" contenuta all'interno delle pagine web, la quale permette uno scope globale degli oggetti inseriti al suo interno e dei loro relativi metodi e variabili.\\

Discutendo con i nostri Tutor sui vantaggi e gli svantaggi che questo approccio presenta, anche se non pienamente consigliato in termini di sicurezza, ci è stato comunque consentito l'utilizzo, in quanto viene già utilizzato in altre applicazioni Loccioni su rete privata e quindi meno soggetta a possibili usi indesiderati, inoltre al momento dello sviluppo, non vi erano altre soluzioni possibili e funzionanti per la comunicazione.\\

Una volta integrata la comunicazione all'interno di Angular abbiamo iniziato la definizione delle API attraverso le quali è possibile lo scambio di informazioni tra le due applicazioni.

\section{Quarta Fase}

Nella quarta ed ultima fase ci siamo dedicati allo sviluppo del prodotto "finale" ovvero la creazione di un servizio Angular utilizzabile all'interno dei componenti o sotto-servizi, per l'integrazione su Dashboard Loccioni del dispositivo Surface Dial.\\

La fase è iniziata modificando radicalmente le API create in precedenza, attraverso lo spostamento di responsabilità che prima erano presenti all'interno dell'UWP, dentro alla parte Web, consentendo così dei messaggi più leggeri nella comunicazione e allo stesso tempo lasciando più libertà ai futuri utilizzatori del servizio, in quanto sino a questo momento, la modifica di un determinato componente avveniva attraverso una dipendenza diretta con il DOM della pagina ricercando l'identificatore dell'elemento da controllare.\\

La modifica ci ha quindi permesso di rimuovere la dipendenza diretta dal DOM e dagli identificatori, permettendo un comportamento dinamico a seconda dell'evento intercettato dal servizio.\\

Alla fine di questa fase abbiamo ricevuto il materiale sul quale effettuare i test per l’utilizzo del Dial su schermo, che ci ha obbligato ad aggiungere funzionalità al servizio, non previste in precedenza, ma consentendoci l'acquisizione di elementi Web della dashboard, attraverso il posizionamento fisico del dispositivo sullo schermo.




