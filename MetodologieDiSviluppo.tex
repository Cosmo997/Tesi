\chapter{Metodologie Di Sviluppo}
\label{chap:metod}

Durante lo sviluppo del progetto abbiamo utilizzato uno sviluppo Scrum unito al modello incrementale, utilizzando un Scrum per quanto riguarda la distribuzione del lavoro e il rapporto con l'azienda, mentre il rilascio incrementale è stato utilizzato una volta scoperte tutte le potenzialità del Dial e il corretto funzionamento su pagine web.

\section{Scrum}
 
Scrum è un framework agile per la gestione del ciclo di sviluppo del software, iterativo ed incrementale, concepito per gestire progetti e prodotti software o applicazioni di sviluppo, creato e sviluppato da Ken Schwaber e Jeff Sutherland.\\

Scrum enfatizza tutti gli aspetti di gestione di progetto legati a contesti in cui è difficile pianificare in anticipo. Vengono utilizzati meccanismi propri di un "processo di controllo empirico", in cui cicli di feedback che ne costituiscono le tecniche di management fondamentali risultano in opposizione alla gestione basata sul concetto tradizionale di command-and-control. Il suo approccio alla pianificazione e gestione di progetti è quello di portare l'autorità decisionale al livello di proprietà e certezze operative.\\

Sono 3 i ruoli che vengono individuati nella metodologia Scrum: \emph{Product Owner}, \emph{Scrum Master} e \emph{Team di Sviluppo}:

\begin{itemize}
\item Il \textbf{Product Owner} definisce il lavoro da svolgere e l'ordine con cui viene completato. Raccoglie la voce degli stakeholder (clienti, management e chiunque abbia un interesse nel prodotto), le necessità dell'utente finale, i requisiti del mercato e sulla base di questi elementi stabilisce le priorità di sviluppo per il Team Scrum.
\item Lo \textbf{Scrum Master} è il responsabile del processo, e un leader a servizio (servant-leader) dello Scrum Team. Conoscitore esperto della metodologia Scrum, sa come applicarla e si assicura che il Team comprenda e segua le regole che la caratterizzano, perché il progetto abbia successo. Inoltre favorisce il lavoro del Team di Sviluppo, rimuovendo ostacoli, organizzando meeting di confronto, e soprattutto proteggendolo da ogni possibile distrazione: ogni membro del gruppo deve poter lavorare al 100 per 100 sullo sviluppo, e lo Scrum Master si assicura che questo avvenga.
\item Il \textbf{Team di Sviluppo} è la squadra di lavoro, composta da 3 a 9 persone. Anche lo Scrum Master può far parte del Team di Sviluppo. Chi concretamente porta a termine gli Sprint e fornisce le funzionalità da implementare è questo insieme coordinato di persone, autogestito e cross-funzionale.
\end{itemize}

Per quanto riguarda il nostro progetto, abbiamo applicato il framework Scrum con i ruoli definiti nel seguente schema:
\begin{itemize}
\item \textbf{Product Owner}: Luca Mazzuferi
\item \textbf{Scrum Master}: Marco Allegrezza, Diego Bonura
\item \textbf{Team di Sviluppo}: Daniele Moschini, Michele Benedetti
\end{itemize}

\newpage
\section{Rilascio Incrementale}

Per \textbf{modello incrementale} si intende, nell'ambito dell'ingegneria del software, un modello di sviluppo di un progetto software basato sulla successione dei seguenti passi principali:

\begin{itemize}
\item Pianificazione
\item Analisi dei requisiti
\item Progetto
\item Implementazione
\item Prove
\item Valutazione
\end{itemize}

Questo ciclo può essere ripetuto diverse volte, in cui ogni "incremento" riduce il rischio di fallimento e produce nuovo valore. Il ciclo viene ripetuto fino a che la valutazione del prodotto diviene soddisfacente rispetto ai requisiti previsti.\\

L'utilizzo del modello incrementale è consigliabile quando si ha, fin dall'inizio della progettazione, una visione abbastanza chiara dell'intero progetto, perché occorre fare in modo che la realizzazione della generica versione k risulti utile per la realizzazione della versione k+1.\\

Un approccio incrementale è particolarmente indicato in tutti quei casi in cui la specifica dei requisiti risulti particolarmente difficoltosa e di difficile stesura (semi)formale. L'uso di questo modello di sviluppo favorisce la creazione di prototipi, ovvero parti di applicazione funzionanti, che a loro volta favoriscono il dialogo con il cliente e la validazione dei requisiti.





