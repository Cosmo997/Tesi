\chapter{SviluppiFuturi}
\label{chap:Capitolo1}

I possibili sviluppi sono molteplici e un’idea iniziale è quella di inserire un nuovo widget all’interno della dashboard Loccioni in grado di raffigurare attraverso un’immagine una parte del motore o elemento da controllare, in modo tale da avere un’interfaccia più intuitiva per la selezione, e poi, una volta selezionato il widget stesso, avere la possibilità di visualizzare i dettagli del componente attraverso il menù del Dial.\\
Una seconda modifica da apportare agli widget esistenti è sicuramente quella del controllo del moltiplicatore da applicare ad ogni rotazione avvenuta con il Dial, portando così maggiore precisione nei lavori più complessi e anche rapidità di utilizzo nelle parti che necessitano più rotazioni di Dial.\\
Uno sviluppo futuro sicuramente necessario è l’aggiunta di un controllo sui dati in input lato web, che momentaneamente non è presente nei nostri widget, ma è fatta solamente lato backend.\\
Per aumentare l’utilizzabilità dell’interfaccia utente è sicuramente necessaria una modifica degli widget da noi creati, per renderli più facilmente selezionabili dal Dial e per migliorare anche l’aspetto visivo.
In caso poi il progetto arrivi al punto di essere utilizzato, per migliorare la manipolazione di dati in uso nelle dashboard attuali è necessario sicuramente apportare alcune modifiche alla dashboard per consentire un utilizzo attraverso il touchscreen di un qualsiasi dispositivo, in modo tale da rendere veramente fluido l’utilizzo di tutto l’ambiente e poter così eliminare il mouse dalle operazioni svolte più frequentemente.\\
Per concludere, un possibile miglioramento da apportare al codice è quello di una selezione più veloce ed accurata del widget attraverso il posizionamento sullo schermo, magari attraverso un movimento anticipato del mouse o un movimento continuo.