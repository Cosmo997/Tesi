\chapter{Conclusioni}
\label{chap: Conclusioni}

Durante l'analisi e lo studio di fattibilitá dell'integrazione di dispositivi HID in ambito aziendale, abbiamo constato che le documentazioni e i possibili problemi paragonabili al nostro erano molto basilari e privi esempi concreti che ci potessero confermare un risultato finale ottimale costringendoci a prolungare la fase di analisi implementando contemporaneamente piccoli prototipi atti alla scoperta effettiva delle potenzialità offerte.\\

Il risultato finale di questo studio conferma che le potenzialità del Surface Dial in ambito industriale sono molteplici, dal miglioramento della User Experience alla precisione adottata nell’utilizzo grazie alle sue caratteristiche principali quali una precisione laser e la facilità di integrazione in contesti differenti.\\

Grazie al servizio sviluppato è stato possibile trasferirle anche in ambito web, sostenuto dal fatto che lo sviluppo e il controllo di software aziendali si sta spostando in maniera decisa verso il web con conseguente portabilità e scalabilità degli applicativi migliorandone l'esperienza utente, ma sopratutto minimizzando tempi e costi.\\


RIVEDERE CONFRONTO VECCHIO NUOVO
\begin{figure}[htpb!]
\center
  \includegraphics[width=0.2\textwidth]{Potenziometri}
  \caption{Potenziometro utilizzato attualmente}
\end{figure}

Nel nostro caso, l’utilizzo finale del Dial nei banchi prova del gruppo Loccioni, comporterebbe una maggior sicurezza in termini di precisione nella variazione di dati, in quanto al momento vengono utilizzati dei semplici potenziometri con le evidenti limitazione che essi comportano, come ad esempio la presenza di un inizio e fine corsa del potenziometro stesso, l’assenza di feedback durante l’utilizzo, la limitata corsa e di conseguenza la necessaria approssimazione dei singoli step eseguiti.

\begin{figure}[htpb!]
\center
\includegraphics[width=0.5\textwidth]{Postazione}
\caption{Postazione banchi attuale}
\end{figure}

Queste problematiche verrebbero risolte attraverso il Dial, in quanto presenta una “corsa” illimitata, un feedback atpico personalizzabile in base alle operazioni che si svolgono, una precisione laser a 3600 punti e la possibilitá di interagire direttamente con lo schermo, appoggiando il Dial su di esso, permettendo quindi una serie di combinazioni di utilizzo estremamente vasta.

\section{Considerazioni}
Terminata la fase di analisi, in concordato con i nostri tutor aziendali, abbiamo scelto tra le varie tecnologie disponibili, un applicazione UWP in quanto ci ha permesso di utilizzare le librerie native del dispositivo Dial e contemporaneamente un supporto costante da parte di Microsoft.\\

A sostegno di ciò, durante la fase di sviluppo, una difficoltà riscontrata è stata l'assenza di una console di debug web all'interno della WebView impedendoci di verificare il corretto funzionamento della comunicazione bidirezionale, ma grazie al continuo rilascio e aggiornamento da parte della community Micosoft di componenti UWP abbiamo a disposizione una nuova versione della webView rilasciata in versione beta a fine Novembre che consente, oltre a varie migliorie, anche l'utilizzo di una console di debug per lo sviluppo.\\

Un' ulteriore difficoltà riscontrata durante la fase di sviluppo nasce dal fatto che la documentazione presente non trattava in maniera specifica l'utilizzo del dispositivo Dial nel nostro contesto.
Le librerie fornite prevedono la sola integrazione con applicazioni native rendendoci, fino a prova contraria, i primi ad integrarlo in un contesto web, mantenendo però tutte le funzionalità native utilizzabili.\\

Un problema riscontrato nell'utilizzo di un'applicazione UWP con una pagina web caricata al suo interno era quello di riuscire ad implementare una comunicazione bidirezionale efficiente tra due linguaggi di programmazione differenti.

Per risolvere questa problematica ci siamo trovati davanti a due possibilitá:

\begin{enumerate}
\item Utilizzare la funzionalitá della WebView chiamata ScriptNotify che permette la notifica all'applicazione UWP dell'avvenuta emissione di un determinato evento nella pagina Web.
\item Utilizzare il decoratore AllowForWeb su una classe, affinché un'istanza di essa, possa essere iniettata nella pagina Web.
\end{enumerate}

Dopo aver letto la documentazione e effettuato vari test, la scelta è ricaduta sul decoratore AllowForWeb, in quanto lo ScriptNotify permetteva di mettersi in ascolto di solamente un determinato evento e permetteva il ritorno di una sola stringa come parametro.
Inoltre, leggendo su varie Community come StackOverflow, questa pratica viene sconsigliata poiché non sicura, in quanto va necessariamente abilitato lato UWP l'URI di ogni singola pagina da controllare e limitandone l'utilizzo in termini di prestazioni.\\ 

In contrapposizione, utilizzare AllowForWeb su di una classe e iniettare un'istanza nella pagina, ci ha permesse un maggior controllo della comunicazione tra Web e UWP grazie all'utilizzo dei metodi messi a disposizione dall'oggetto, permettendoci di scambiare maggiori informazioni e parametri tipati e non solamente stringhe.\\

\section{Sviluppi Futuri}

I possibili sviluppi sono molteplici e un’idea iniziale è quella di inserire un nuovo widget all’interno della dashboard Loccioni in grado di raffigurare attraverso un’immagine, una parte del motore o un componente da controllare, in modo tale da avere un’interfaccia più intuitiva per la selezione, e successivamente, una volta acquisito il widget stesso, avere la possibilità di visualizzare i dettagli e le funzionalità del componente attraverso il menù del Dial.\\

Uno sviluppo futuro sicuramente necessario è l’aggiunta di un controllo sui dati in input lato web, che momentaneamente non è presente nei nostri widget, ma è fatta solamente lato backend.\\

Per aumentare l'utilizzabilità dell’interfaccia utente è sicuramente necessaria una modifica degli widget da noi creati, per renderli più facilmente selezionabili dal Dial e per migliorare anche l’aspetto visivo.

Il servizio implementato presenta una violazione del principio di singola responsabilità in quanto oltre a gestire la comunicazione bidirezionale con l'applicazione UWP, si occupa anche di tenere traccia di quale Widget presente nella Dashboard é stato acquisito dal Dial in fase di RunMode, escludendo l'ascolto dell'invoke sui suoi canali.
Pertanto, abbiamo pianificato di spostare questa seconda responsabilitá, creando un nuovo servizio dedicato a questo compito.

Qualora il progetto arrivi al punto di essere utilizzato e rilasciato in ambienti industriali, per migliorare la manipolazione di dati modificabili nelle dashboard attuali sará necessario apportare alcune modifiche alla dashboard, per consentire un utilizzo attraverso il touchscreen di un qualsiasi dispositivo, in modo tale da rendere veramente fluido l’utilizzo di tutto l’ambiente e poter così eliminare il mouse dalle operazioni svolte più frequentemente o che richiedono una maggior precisione nella variazione dei dati.\\



\section{Evoluzioni dello State of Art}


Al momento dello sviluppo, la necessitá di dover passare per un applicazione UWP rappresenta sicuramente una limitazione che andrá tolta qualora il progetto verrá rilasciato in ambito industriale.\\

A favore di questa considerazione, nel mese passato, la Google stessa ha rilasciato in fase di Preview una libreria Web chiamata Trial for WebHID, la quale permetterá di far comunicare direttamente con la pagina Web i dispositivi che utilizzano il protocollo HID, come il Dial, permettendo agli sviluppatori futuri di non dover necessariamente passare attraverso una WebView o una applicazione nativa per quel dispositivo.\\

-HID nel web utilizzo
-Cloud game -> utilizzo potenza di calcolo esterna all'hardware
-Microsoft spinge sul web con webview2
-portabilità






