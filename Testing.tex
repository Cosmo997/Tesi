\chapter{Testing}
\label{chap:Capitolo1}

Per quanto riguarda i test svolti in ambito UWP abbiamo utilizzato MSTest fornito da Microsoft all’interno di Visual Studio per testare le funzionalitá messe a disposizione dalla classe RadialController.

\section{MSTest}
Visual Studio Unit Testing Framework descrive la suite di strumenti di unit test di Microsoft integrata in alcune versioni di Visual Studio 2005 e successive.
Qui sotto sono riportati due tra i vari test implementati nel progetto:
 
\vspace{1.0cm}
\begin{lstlisting}[caption={Test SetMenu},style=javaScriptCode]
    [UITestMethod]
        public void TestSetMenu()
        {
            Assert.IsFalse(radialController.Menu.Items.Count > 0);
            DialMenuItem item1 = dialController.CreateDialMenuItem("tag1", "d", "icon");
            DialMenuItem item2 = dialController.CreateDialMenuItem("tag2", "d", "icon");
            DialMenuItem item3 = dialController.CreateDialMenuItem("tag3", "d", "icon");
            DialMenuItem item4 = dialController.CreateDialMenuItem("tag4", "d", "icon");
            List<DialMenuItem> items = new List<DialMenuItem>();
            items.Add(item1);
            items.Add(item2);
            items.Add(item3);
            items.Add(item4);
            dialController.SetMenu(items);
            Assert.IsTrue(radialController.Menu.Items.Count == 4);
        }
\end{lstlisting} 
\vspace{1.0cm}

\vspace{1.0cm}
\begin{lstlisting}[caption={Test aggiunta voce di Menu'},style=javaScriptCode]
  [UITestMethod]
        public void TestAddItem(){
            Assert.IsFalse(radialController.Menu.Items.Count > 0);
            dialController.AddItem(dialController.CreateDialMenuItem("tag", "d", "icon"));
            Assert.IsTrue(radialController.Menu.Items.Count > 0);
        }
\end{lstlisting} 
\vspace{1.0cm}
